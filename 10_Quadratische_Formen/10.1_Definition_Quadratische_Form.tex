\subsection{10.1 Definition Quadratische Form}{
\vskip2pt
Quadratische Formen sind bestimmte Funktionen, die mit einer symmetrischen Matrix $A$ und einem Vektor $x \in \mathbb{R}^n$ gebildet werden. Es kommen maximal quadratische Terme vor.

\begin{center}
\fbox{$q_a(x_1, x_2, \hdots, x_n) = q(x) = x^TAx$}
\end{center}

\begin{itemize}[leftmargin=0.29cm, itemsep=0pt]
\item Quadratische Formen, die mit einer diagonalen Matrix gebildet werden (siehe $q_c$) nennt man \textbf{rein quadratisch}.
\item Die symmetrische Matrix $A$ legt die Gestalt der entstehenden Fläche fest.
\end{itemize}


\vspace{2pt}

\textbf{Beispiele:} \par
Für $\mathbb{R}^2$:

$q_A(\underline{x}) = (x_1 \hskip2pt x_2 \hskip2pt x_3)  \begin{pmatrix} a & d\\ d & b\end{pmatrix}  \begin{pmatrix} x_1 \\ x_2\end{pmatrix}\\
\hphantom{q_A(\underline{x})} = ax_1^2 + 2dx_1x_2 + bx_2^2$\vskip2pt

Für $\mathbb{R}^3$:
\vspace{-11pt}

\begin{align}
q_A(\underline{x}) &= (x_1 \hskip2pt x_2 \hskip2pt x_3)  \begin{pmatrix} a & d & e \\ d & b & f \\ e & f & c \end{pmatrix}  \begin{pmatrix} x_1 \\ x_2 \\ x_3 \end{pmatrix} \nonumber\\ 
&= ax_1^2 + bx_2^2 + cx_3^2 + 2dx_1x_2 + 2ex_1x_3 + 2fx_2x_3 \nonumber
\end{align}

}
\WhiteSpace
