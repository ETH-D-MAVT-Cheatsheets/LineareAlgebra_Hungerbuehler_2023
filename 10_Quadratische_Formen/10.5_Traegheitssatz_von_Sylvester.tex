\subsection{10.5 Traegheitssatz von Sylvester}{
\vskip1pt

\textbf{Signatur} \vskip1pt
Die Signatur einer Matrix $A$ ist definiert als:

\vspace{-2pt}
\begin{center}
\fbox{$sig(A) = (p,n,z)$}
\end{center}
\par
\vspace{-2pt}

Wobei
\vspace{-4pt}
\begin{itemize}[leftmargin=0.29cm, itemsep=0pt]
\item $A \in \mathbb{R}^{m \times m}$ symmetrisch ist.
\item $p$ die Anzahl positiver EW von $A$.
\item $n$ die Anzahl negativer EW von $A$ (je mit algVh gezählt).
\item $z = m-p-n$ die algVh des EW 0 ist.
\end{itemize}
\vskip2pt

\textbf{Traegheitssatz von Sylvester}\vskip2pt
Ist $A \in \mathbb{R}^{m \times m}$ symmetrisch und $W \in \mathbb{R}^{m \times m}$ regulär so haben $A$ und $W^TAW$ die selbe Signatur. Des Weiteren existiert eine Matrix $W$ so dass

\vspace{-2pt}
\begin{center}
$W^TAW = diag(\tikzmark[xshift=0,yshift=-6pt]{p_1} 1,...,1 \tikzmark[xshift=0,yshift=-6pt]{p_2}, \tikzmark[xshift=0,yshift=-6pt]{n_1} -1,...,-1 \tikzmark[xshift=0,yshift=-6pt]{n_2}, \tikzmark[xshift=0,yshift=-6pt]{z_1} 0,...,0 \tikzmark[xshift=0,yshift=-6pt]{z_2})$
\end{center}
\par\vskip4pt

\drawbrace[brace mirrored, thick, black]{p_1}{p_2}
\drawbrace[brace mirrored, thick, black]{n_1}{n_2}
\drawbrace[brace mirrored, thick, black]{z_1}{z_2}
\annote[below=3pt, black]{brace-7}{\small $p$}
\annote[below=3pt, black]{brace-8}{\small $n$}
\annote[below=3pt, black]{brace-9}{\small $z$}

}
\WhiteSpace