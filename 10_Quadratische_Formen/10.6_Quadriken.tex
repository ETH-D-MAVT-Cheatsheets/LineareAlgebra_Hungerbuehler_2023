\subsection{10.6 Quadriken}{
\vskip1pt
Setzt man eine quadratische Form in eine Gleichung folgender Form ein ($x \in \mathbb{R}^n, a \in \mathbb{R}^n, b \in \mathbb{R}$), erhält man eine sogenannte Quadrik:

\begin{center} $q(x) + a^Tx + b = 1 $\end{center}


Ist die quadratische Form zweidimensional, erhält man einen sogenannten Kegelschnitt, ist sie dreidimensional erhält man eine Fläche zweiten Grades.
}
\colbreak
%\WhiteSpace