\subsection{12.4 Bedingungen im Unendlichen}{
\vskip1pt

Für das Bestimmen der Konstanten $C_i$ sind nicht immer nur Anfangsbedingungen $y_i(0)$ gegeben, sondern manchmal auch Bedingungen wie $\lim\limits_{t \rightarrow \infty}{y_i(t)} = a$. \vskip6pt

Bsp: \hskip6pt Bestimme $C_i$ von $y(t) = C_1 + 3C_2e^{-t} + C_3\cdot e^{2t}$ \vskip2pt
\hskip22pt mit $y(0) = 2$ und $\lim\limits_{t \rightarrow \infty}{y(t)} = 5$

\vskip5pt

\textbf{Vorgehen:} \vskip1pt

\begin{enumerate}[label=\protect\circled{\arabic*}]
\item Verlangt eine Bedingung, dass $y(t)$ im Unendlichen beschränkt sein soll, setze Konstanten vor Exponentialfunktionen mit positiven Exponenten null.
\vskip2pt Bsp: \scalebox{0.9}{\hskip8pt Zweite Bedingung $\lim\limits_{t \rightarrow \infty}{y(t)} = 5 \hskip3pt \Rightarrow \hskip3pt C_3 = 0$} \par

\item Man bestimme weitere Konstanten, indem man $t \rightarrow \infty$ einsetzt.
\vskip2pt Bsp: \scalebox{0.9}{\hskip8pt$\lim\limits_{t \rightarrow \infty}{y(t)} = C_1 = 5$} \par

\item Man bestimme die übrigen Konstanten, indem man $t = 0$ einsetzt.
\vskip2pt Bsp: \scalebox{0.9}{\hskip8pt $ y(0) = 5 + 3 C_2 = 2 \hskip3pt \Rightarrow \hskip3pt C_2 = -1$} \par
\end{enumerate}

}
\WhiteSpace