\subsection{6.6 Diagonalisierbarkeit}{
\vskip1pt
Eine quadratische Matrix A heisst diagonalisierbar, falls eine reguläre Matrix T existiert, sodass $D = T^{-1}AT$ eine Diagonalmatrix ist. 
\par

\begin{center}
\fbox{$A$ halbeinfach $\Longleftrightarrow A$ diagonalisierbar}
\end{center}

\textbf{Matrix diagonalisieren} (Basiswechsel in Eigenbasis)
\begin{enumerate}[label=\protect\circled{\arabic*}]
\item Bestimme die Eigenwerte $\lambda_i$ und die Eigenvektoren $v_i$
\item Die Matrix $D = diag(\lambda_1, \hdots, \lambda_n)$ ist eine Diagonalmatrix mit den Eigenwerten auf der Diagonalen.
\item Die Matrix $T = (v_1, \hdots, v_n)$ hat die Eigenvektoren als Spalten \textbf{(Gleiche Reihenfolge wie bei D!)}.
\item Bestimme $T^{-1}$. Falls EV orthonormal $T^{-1} = T^T$
\end{enumerate}

\vskip3pt

\textbf{Potenzen und Exponentialfunktion} \par \vskip1pt
Potenzen/Exponentialfunktionen von diagonalisierbaren Matrizen können einfach berechnet werden:
\begin{itemize}[leftmargin=0.29cm, itemsep=0.5pt]
\item $A^k = (TDT^{-1})^k = T\cdot diag(\lambda_1^k, \hdots, \lambda_n^k)\cdot T^{-1}$
\item $e^A = e^{TDT^{-1}} = T\cdot diag(e^{\lambda_1}, \hdots, e^{\lambda_n})\cdot T^{-1}$
\end{itemize}


}
\WhiteSpace