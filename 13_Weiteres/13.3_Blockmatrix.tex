\subsection{13.3 Blockmatrix}{
\vskip1pt

Eine Blockmatrix ist eine Matrix welche in mehrere \grqq{}Blöcke\grqq{} unterteilt werden kann.\par

Beispiel einer $2 \times 2$ Blockmatrix:
\begin{center}
$P = \begin{bmatrix}
			\pmb A & \pmb B \\
			\pmb C & \pmb D
	 \end{bmatrix}$
\end{center}

\textbf{Determinante}\par\vskip1pt
Für die Determinante einer $2 \times 2$ Blockmatrix mit einem 0 Block gilt die Eigenschaft\par\vspace{-4pt}
\begin{center}
$det\begin{pmatrix} \pmb A & 0\\	\pmb C & \pmb D\end{pmatrix} = det(\pmb A)det(\pmb B)$
\end{center}\par\vskip3pt

Handelt es sich bei den Blöcken um quadratische Matrizen der gleichen Größe, welche kommutieren ($CD = DC$) dann gilt des Weiteren\par\vspace{-4pt}
\begin{center}
$det\begin{pmatrix} \pmb A & \pmb B\\ \pmb C & \pmb D\end{pmatrix} = det(\pmb{AD} - \pmb{BC})$
\end{center}\par\vskip3pt

Und falls $A=D$ und $B=C$ gilt\par\vspace{-4pt}
\begin{center}
$det\begin{pmatrix} \pmb A & \pmb B\\ \pmb B & \pmb A\end{pmatrix} = det(\pmb A - \pmb B)det(\pmb A + \pmb B)$
\end{center}\par\vskip3pt


\textbf{Eigenwerte}\par\vskip1pt
Die Eigenwerte einer $2 \times 2$ Blockmatrix $P$ mit einem 0 Block ergeben sich aus\par\vspace{-4pt}
\begin{center}
$det\begin{pmatrix} \pmb A - \lambda\pmb I & 0\\ \pmb C & \pmb D - \lambda\pmb I\end{pmatrix} = det(\pmb A - \lambda\pmb I)det(\pmb D - \lambda\pmb I)$
\end{center}\par\vskip3pt

Deshalb gilt $eig(P) = (eig(A), eig(D))$

}
\WhiteSpace