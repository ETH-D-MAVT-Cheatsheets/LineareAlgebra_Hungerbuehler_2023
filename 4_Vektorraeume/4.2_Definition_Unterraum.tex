\subsection{4.2 Definition Unterraum}{
\vskip1pt
Eine nichtleere Teilmenge eines Vektorraums V heisst Unterraum von V, falls:
\vskip6pt
\begin{enumerate}[label=\protect\circled{\arabic*}]
\item \hskip5pt $\forall a, b \in U:$ \hskip29pt $a \bigoplus b \in U$
\item \hskip5pt $\forall a \in U, \forall \alpha \in \mathbb{K}:$ \hskip10pt$ \alpha \bigodot a \in U$
\end{enumerate}

\begin{itemize}[leftmargin=0.29cm, itemsep=0.5pt]
\item Ein Unterraum ist selber ein Vektorraum.
\item Ein Unterraum \textbf{muss den Nullvektor enthalten!}
\end{itemize}
\vspace{-1.5mm}
}
\WhiteSpace