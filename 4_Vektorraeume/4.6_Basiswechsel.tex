\subsection{4.6 Basiswechsel}{
\vskip1pt

Sei $V^n$ ein Vektorraum mit Basen $Q = \{q_1, q_2, \hdots, q_n\}$ und $W = \{w_1, w_2, \hdots, w_n\}$. Sei $v$ ein Vektor $\in V$.
\par
\vskip8pt

\textbf{Basiswechsel von $[v]_q$ nach $[v]_w$ durchführen:}
\begin{enumerate}[label=\protect\circled{\arabic*}]
\item Übergangsmatrix: $T_{q \rightarrow w} = ([q_1]_w, \hdots, [q_n]_w)$
\item $[v]_w = T_{q \rightarrow w}\cdot[v]_q$
\end{enumerate}
\vskip6pt

\textbf{Tipps:}
\begin{itemize}[leftmargin=0.29cm, itemsep=0.5pt]
\item $T_{w \rightarrow q} = T_{q \rightarrow w}^{-1}$
\item Meist ist eine der beiden Basen die Standardbasis $S$. Die Übergangsmatrix $T_{q \rightarrow s}$ ist dann sehr einfach bestimmbar. Die entegengesetzte Übergangsmatrix wird am schnellsten durch invertieren gefunden.
\item Basiswechsel für Matrizen: \par $[A]_w = T_{q \rightarrow w}\cdot [A]_q \cdot T_{q \rightarrow w}^{-1}$
\item Falls $T$ orthogonal: $T^{-1} = T^T$
\end{itemize}

\vspace{-3pt}
}
\WhiteSpace