\subsection{4.1 Definition Vektorraum}{
\vskip1pt
Sei V eine Menge von Objekten. V heisst Vektorraum, wenn eine \textbf{innere Operation} (Kombination von zwei Objekten) und eine \textbf{äussere Operation} (Kombination eines Objekts mit einem Skalar) definiert sind, und folgende Axiome gelten:

\vskip3pt
\hskip6pt
\begin{minipage}[c]{0.35 \columnwidth}
\textbf{Innere Operation:}
\vskip1pt
$\bigoplus$: \hskip2pt $V \times V \rightarrow V$ \par
\hskip15pt $(a, b) \mapsto a \bigoplus b$
\end{minipage}
\hskip30pt
\begin{minipage}[c]{0.35 \columnwidth}
\textbf{Äussere Operation:}
\vskip1pt
$\bigodot$: \hskip2pt $\mathbb{K} \times V \rightarrow V$ \par
\hskip15pt $(\alpha, a) \mapsto \alpha \bigodot a$
\end{minipage}

\vskip3pt
\hskip6pt
\textbf{Axiome:} \vskip2pt \par
\hskip6pt
\begin{minipage}[t]{0.36 \columnwidth}
\parskip8pt
(A1) $\forall u, v \in V:$ \par
(A2) $\forall u, v, w \in V:$ \par
(A3) $\exists 0 \in V,$ \par \vspace{-4pt}
\hskip17pt $\forall u \in V:$ \par
(A4) $\forall u \in V,$ \par \vspace{-4pt}
\hskip17pt $\exists -u \in V:$
\end{minipage}
\begin{minipage}[t]{0.6 \columnwidth}
\parskip8pt
$u \bigoplus v = v \bigoplus u$ \par
$(u \bigoplus v) \bigoplus w = u \bigoplus (v \bigoplus w)$ \par
$u \bigoplus 0 = u$ \par \vskip12pt
$u \bigoplus (-u) = 0$
\end{minipage}

\vskip12pt
\hskip6pt
\begin{minipage}[t]{0.36 \columnwidth}
\parskip8pt
(M1) $\forall \alpha, \beta \in \mathbb{R},$ \par \vspace{-4pt}
\hskip17pt $\forall u \in V:$ \par
(M2) $\forall \alpha, \beta \in \mathbb{R},$ \par \vspace{-4pt}
\hskip17pt $\forall u, v \in V:$ \par
(M3) $\forall u \in V:$
\end{minipage}
\begin{minipage}[t]{0.6 \columnwidth}
\parskip8pt
$(\alpha \cdot \beta) \bigodot u = \alpha \bigodot (\beta \bigodot u)$ \par \vskip12pt
$(\alpha + \beta) \bigodot u = (\alpha \bigodot u) \bigoplus (\beta \bigodot u)$ \par \vskip-4pt
$\alpha \bigodot (u \bigoplus v) = (\alpha \bigodot u) \bigoplus (\alpha \bigodot v)$
\vskip0pt
$1 \bigodot u = u$

\end{minipage}

}
\WhiteSpace