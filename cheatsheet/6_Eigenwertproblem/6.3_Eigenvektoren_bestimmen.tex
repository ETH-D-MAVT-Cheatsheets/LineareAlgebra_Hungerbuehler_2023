\subsection{6.3 Eigenvektoren bestimmen}{
\vspace{-1pt}
Nach dem Bestimmen der Eigenwerte können die zugehörigen Eigenvektoren bestimmt werden.
\begin{enumerate}[label=\protect\circled{\arabic*}, itemsep=-1pt]
\item Setze einen Eigenwert $\lambda_k$ in $(A - \lambda_k \cdot I)$ ein
\item Löse das Gleichungssystem $(A - \lambda_k \cdot I)\cdot x = 0$
\item Das Gleichungssystem hat unendlich viele Lösungen. Man erhält einen oder mehrere mit freien Parametern multiplizierte Eigenvektoren $v_k$.
\end{enumerate}

\vskip1pt

\textbf{Eigenschaften}
\vspace{-2pt}
\begin{itemize}[leftmargin=0.29cm, itemsep=0pt]
\item Eigenvektoren sind per Definition $\neq 0$
\item Eigenvektoren sind linear unabhängig.
\item Komplex konjugierte Eigenwerte haben komplex konjugierte Eigenvektoren (spart Zeit bei Berechnung).
\item $Av = \lambda v \rightarrow A^{n-1}(Av) = A^{n-1}(\lambda v) = \lambda^nv$
\end{itemize}
\vspace{-4pt}

}
\WhiteSpace