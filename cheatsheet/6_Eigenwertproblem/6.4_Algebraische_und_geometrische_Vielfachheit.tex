\subsection{6.4 Algebraische und geometrische Vielfachheit}{

\begin{center}
\fbox{$1 \leq$ gVfh. von $\lambda \leq$ algVfh. von $\lambda \leq n$}
\end{center}
\vspace{3pt}

\textbf{Algebraische Vielfachheit} \par \vskip1pt
Die algebraische Vielfachheit ist die Vielfachheit einer Nullstelle im charakteristischen Polynom $p(\lambda)$ beim jeweiligen Eigenwert $\lambda$. \par
\vskip4pt
\hskip6pt \textbf{Bsp:} $p(\lambda) = (\lambda - 3)^2 \cdot (\lambda - 2)$ \par \vskip1pt
\hskip17pt $\Longrightarrow$ $\lambda = 3$ hat algVfh. 2 und $\lambda = 2$ hat algVfh. 1

\vskip3pt

\textbf{Geometrische Vielfachheit} \par \vskip1pt
Die geometrische Vielfachheit von $\lambda$ ist die Anzahl der zum EW gehörigen EV = Anzahl der freien Parameter.
}
\WhiteSpace