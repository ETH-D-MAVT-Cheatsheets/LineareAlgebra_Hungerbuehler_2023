\subsection{10.4 Extrema einer quadratischen Form}{
\vskip1pt
\textbf{Kritische Punkte finden:} \vskip2pt
Man setze den Gradienten der quadratischen Form $grad(q(x)) = (\frac{dq}{dx_1}, \frac{dq}{dx_2}, \hdots, \frac{dq}{dx_n})^T = 0$. Durch Lösen des Gleichungssystems erhält man die Koordinaten der kritischen Punkte.

\vskip5pt

\textbf{Kritische Punkte zuordnen:} \vskip1pt

\begin{enumerate}[label=\protect\circled{\arabic*}]
\item Bilde Hessesche Matrix in der richtigen Dimension für jeden kritischen Punkt: \par
Bsp: \hskip10pt$ H_{2x2} = \begin{pmatrix} \frac{dq(x)^2}{d^2 x_1} & \frac{dq(x)^2}{dx_1x_2} \\ \frac{dq(x)^2}{dx_1x_2} & \frac{dq(x)^2}{d^2 x_2} \end{pmatrix}$
\item Bestimme Definitheit der Matrix (siehe 9.3). \vskip2pt
positiv definit $\Longrightarrow$ lokales Minimum\par
negativ definit $\Longrightarrow$ lokales Maximum\par
indefinit $\Longrightarrow$ Sattelpunkt

\end{enumerate}
\vspace{-4pt}
}
\WhiteSpace