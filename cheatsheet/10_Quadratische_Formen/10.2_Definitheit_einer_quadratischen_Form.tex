\subsection{10.2 Definitheit einer quadratischen Form}{
\vskip1pt

Eine quadratische Form heisst:

\vspace{-4pt}
\begin{center}
\begin{minipage}[c]{0.8 \columnwidth}
\begin{itemize}[leftmargin=0.29cm, itemsep=0pt]
\item positiv definit: \hskip27pt $q(x) > 0 \hskip2pt \forall x \neq 0$
\item negativ definit: \hskip25pt $q(x) < 0 \hskip2pt\forall x \neq 0$
\item positiv semidefinit: \hskip13pt $q(x) \geq 0 \hskip2pt \forall x \neq 0$
\item negativ semidefinit: \hskip11pt $q(x) \leq 0 \hskip2pt \forall x \neq 0$
\item indefinit: \hskip45pt sonst
\end{itemize}
\end{minipage}
\end{center}

Um die Definitheit einer quadratische Form zu bestimmen, bestimme man die Definitheit der zugehörigen symmetrischen Matrix $A$ (siehe 9.3)

}
\WhiteSpace