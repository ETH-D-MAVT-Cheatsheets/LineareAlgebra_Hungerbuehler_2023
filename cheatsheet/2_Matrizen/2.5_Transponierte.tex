\subsection{2.5 Transponierte}{
\vskip1pt
Die Transponierte einer Matrix erhält man, indem man sie an ihrer Diagonalen \glqq spiegelt\grqq.

\vspace{0.5mm}

\begin{center}

\textbf{Bsp:} 
$\begin{pmatrix}
a & b \\
c & d \\
e & f \\
\end{pmatrix}^T
=
\begin{pmatrix}
a & c & e \\
b & d & f \\
\end{pmatrix}$

\end{center}

\textbf{Rechenregeln:} \par
\vspace{-1.5mm}
\begin{minipage}[t]{0.49 \columnwidth}
\begin{align}
(A + B)^T &= A^T + B^T \nonumber \\
(A \cdot B)^T &= B^T \cdot A^T \nonumber \\
(c \cdot A)^T &= c \cdot A^T \nonumber \\
(A^T)^T &= A \nonumber
\end{align}
\end{minipage}
\begin{minipage}[t]{0.49 \columnwidth}
\begin{align}
(A^T)^{-1} &= (A^{-1})^T \nonumber \\
rang(A^T) &= rang(A) \nonumber \\
det(A^T) &= det(A) \nonumber \\
eig(A^T) &= eig(A) \nonumber
\end{align}
\end{minipage}


}
\WhiteSpace