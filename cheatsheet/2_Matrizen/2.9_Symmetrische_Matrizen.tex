\subsection{2.9 Symmetrische Matrizen}{
\vskip1pt

\textbf{Symmetrische Matrix}\vskip2pt
Eine symmetrische Matrix ist eine quadratische Matrix, deren Einträge spiegelsymmetrisch bezüglich der Hauptdiagonalen sind. \par
Dies ist der Fall, wenn sie gleich ihrer Transponierten ist:


\begin{center}
\fbox{$S = S^T$}
\end{center}


\textbf{Eigenschaften:}
\vspace{-3pt}

\begin{itemize}[leftmargin=0.29cm, itemsep=0.5pt]
\item $A^T A$ und $A A^T$ sind immer symmetrisch.
\item Die Eigenwerte von $S$ sind alle reell.
\item Ist $x$ ein Eigenvektor von $S$ zum Eigenwert $\lambda$, so sind auch $konj(x)$, $Re(x)$, $Im(x)$ Eigenvektoren zum selben Eigenwert $\lambda$
\item Die Eigenvektoren zu unterschiedlichen Eigenwerten sind orthogonal zueinander.
\item $S$ ist halbeinfach, also diagonalisierbar.
\item $S$ besitzt eine orthonormale Eigenbasis.
\item Transformationsmatrix $T$ in Eigenbasis kann orthogonal gewählt werden.
\item $S = (s_{ij}) \in \mathbb{R}^{n \times n}$ mit $s_{ij} = s_{ji}$
\end{itemize}\vskip4pt

\textbf{Schiefsymmetrische Matrix}\vskip2pt
Eine schiefsymmetrische Matrix ist eine quadratische Matrix, welche gleich dem Negativen ihrer Transponierten ist:


\begin{center}
\fbox{$-S = S^T$}
\end{center}


\textbf{Eigenschaften:}
\vspace{-3pt}

\begin{itemize}[leftmargin=0.29cm, itemsep=0.5pt]
\item Die Eigenwert von $S$ sind alle imaginär oder gleich 0
\item Alle Diagonaleinträge sind notwendigerweise gleich 0
\item $det(S) = det(S^T) = det(-S) = (-1)^n det(S)$
\item $S = (s_{ij}) \in \mathbb{R}^{n \times n}$ mit $s_{ij} = -s_{ji}$
\end{itemize}
\vspace{-2mm}

}
\WhiteSpace