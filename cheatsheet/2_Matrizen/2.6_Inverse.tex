\subsection{2.6 Inverse}{

Die Inverse $A^{-1}$ von $A$ macht eine Multiplikation mit $A$ rückgängig. Multipliziert man $A$ mit $A^{-1}$, erhält man die Identitätsmatrix.

\vspace{0pt}

\begin{center}
\fbox{\parbox[c][0.7cm][t]{0.55\columnwidth }{\centering $A\cdot v = w \Leftrightarrow A^{-1}\cdot w = v$ 
\\\addvspace{3pt}
$A^{-1} \cdot A = A \cdot A^{-1} = I$}}
\end{center}

\textbf{Eigenschaften:}

 \vspace{-3pt}

\begin{itemize}[leftmargin=0.29cm, itemsep=0pt]
\item Nur quadratische Matrizen können invertierbar sein.
\item Eine invertierbare Matrix nennt man \textbf{regulär}, eine nicht invertierbare \textbf{singulär}.
\item Die Inverse ist eindeutig.
\item $A$ ist invertierbar $\Longleftrightarrow$ $A$ hat vollen Rang
\item $A$ ist invertierbar $\Longleftrightarrow$ $A^T$ ist invertierbar
\item $A$ ist symmetrisch $\Longleftrightarrow$ $A^{-1}$ ist symmetrisch
\item $A$ ist eine Dreiecksmatrix $\Longleftrightarrow$ $A^{-1}$ ist eine Dreiecksmatrix
\item $A$ ist invertierbar $\Longleftrightarrow$ $det(A) \neq 0$
\item $A$ ist invertierbar $\Longleftrightarrow$ kein Eigenwert $\lambda = 0$
\item $A$ und $B$ sind invertierbar $\Longrightarrow$ $AB$ ist invertierbar
\end{itemize}

\vskip0mm

\textbf{Gauss-Jordan Algorithmus:}\par
\vspace{1mm}
Methode zur Bestimmung der Inversen. Man schreibt die Matrix und die Identität nebeneinander auf und führt den Gaussalgorithmus gleich auf beiden Seiten aus, sodass am Ende auf der linken Seite die Identitätsmatrix steht. \par
\textbf{Tipp:} Erzeuge zuerst durch \glqq nach unten gaussen\grqq \hskip2pt links eine Rechtsdreiecksmatrix, dann durch \glqq nach oben gaussen\grqq  \hskip2pt eine Diagonalmatrix, und am Ende durch Zeilenmultiplikation die Identitätsmatrix.

\vskip1.5mm

\scalebox{0.97}{
\hspace{-2mm}
\begin{minipage}[c]{0.44 \columnwidth}
$\begin{pmatrix}
1 & 2 & 0 & \multicolumn{1}{|c}{1} & 0 & 0 \\
2 & 3 & 0 & \multicolumn{1}{|c}{0} & 1 & 0 \\
\tikzmark[xshift=0,yshift=-6pt]{i_1} 3 & 4 & 1 \tikzmark[xshift=0,yshift=-6pt]{i_2} & \multicolumn{1}{|c}{0} & 0 & 1 \\
\end{pmatrix}$
\end{minipage}
\begin{minipage}[t]{0.041 \columnwidth}
\begin{center}
$\Rightarrow$
\end{center}
\end{minipage}
\begin{minipage}[c]{0.38 \columnwidth}
$\begin{pmatrix}
1 & 0 & 0 & \multicolumn{1}{|c}{-3} & 2 & 0 \\
0 & 1 & 0 & \multicolumn{1}{|c}{2} & -1 & 0 \\
0 & 0 & 1 & \multicolumn{1}{|c}{\tikzmark[xshift=0,yshift=-6pt]{i_3} 1} & -2 & 1 \tikzmark[xshift=0,yshift=-6pt]{i_4} \\
\end{pmatrix}$
\end{minipage}

\drawbrace[brace mirrored, thick, blue!50!black]{i_1}{i_2}
\drawbrace[brace mirrored, thick, red!80!black]{i_3}{i_4}
\annote[below=3pt, blue!50!black]{brace-5}{\small $A$}
\annote[below=3pt, red!80!black]{brace-6}{\small $A^{-1}$}
}

\vskip5mm

\textbf{Adjunktenformel für 2x2-Matrizen:}
\vspace{2pt}

\begin{center}
$\begin{pmatrix}
a & b \\
c & d 
\end{pmatrix}^{-1}
=
\dfrac{1}{ad - bc} \cdot
\begin{pmatrix}
d & -b \\
-c & a 
\end{pmatrix}$
\end{center}

\vskip0mm

\textbf{Rechenregeln:} \par
\vspace{-2.5mm}
\begin{minipage}[t]{0.49 \columnwidth}
\begin{align}
I^{-1} &= I\nonumber \\
(A^{-1})^{-1} &= A \nonumber \\
(A^k)^{-1} &= (A^{-1})^k \nonumber \\
(c\cdot A)^{-1} &= c^{-1} \cdot A^{-1} \nonumber \\
(A\cdot B)^{-1} &= B^{-1} \cdot A^{-1} \nonumber
\end{align}
\end{minipage}
\begin{minipage}[t]{0.49 \columnwidth}
\begin{align}
(A^T)^{-1} &= (A^{-1})^T \nonumber \\
rang(A^{-1}) &= rang(A) \nonumber \\
det(A^{-1}) &= det(A)^{-1} \nonumber \\
eig(A^{-1} &= eig(A)^{-1} \nonumber 
\end{align}
\end{minipage}

}