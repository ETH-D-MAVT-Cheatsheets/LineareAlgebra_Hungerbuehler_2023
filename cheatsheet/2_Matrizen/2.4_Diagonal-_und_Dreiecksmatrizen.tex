\subsection{2.4 Diagonal- und Dreiecksmatrizen}{

\vskip1pt

Eine \textbf{Diagonalmatrix} ist eine quadratische Matrix, deren Elemente ausserhalb der Hauptdiagonalen 0 sind.

\begin{center}

$ D = diag(d_1, d_2, d_3) = \begin{pmatrix} d_1 & 0 & 0 \\ 0 & d_2 & 0 \\ 0 & 0 & d_3 \end{pmatrix}$

\end{center}

Eine \textbf{Dreiecksmatrix} ist eine quadratische Matrix, deren Elemente entweder oberhalb oder unterhalb der Hauptdiagonalen Null sind. Man unterscheidet zwischen einer \textbf{Rechtsdreiecksmatrix} und einer \textbf{Linksdreiecksmatrix.}

\begin{center}

$ R = \begin{pmatrix} r_{11} & r_{12} & r_{13} \\ 0 & r_{22} & r_{23} \\ 0 & 0 & r_{33} \end{pmatrix}$
\hskip10pt
$ L = \begin{pmatrix} l_{11} & 0 & 0 \\ l_{21} & l_{22} & 0 \\ l_{31} & l_{32} & l_{33} \end{pmatrix}$

\end{center}

\textbf{Für Diagonal- und Dreiecksmatrizen gilt:} \par
\begin{itemize}[leftmargin=0.29cm, itemsep=0.8pt]
\item $det(A) = a_{11} \cdot a_{22} \cdot a_{33}   \dotsm a_{nn}$
\item $eig(A) = \{a_{11}, a_{22}, \cdots ,  a_{nn}\}$
\end{itemize}

}