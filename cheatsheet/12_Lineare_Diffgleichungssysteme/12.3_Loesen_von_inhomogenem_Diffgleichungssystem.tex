\subsection{12.3 Lösen von inhomogenem Diff'gleichungssystem}{
\vskip1pt

Man hat bereits mit dem in 11.1 beschriebenen Verfahren die Lösung $y_h(t)$ für das homogene Diff'gleichungssystem $y' = A\cdot y$ gefunden. Jetzt sucht man die Lösung für das inhomogene System: \vskip5pt
\begin{center}
$y' = A\cdot y + b$: \vskip5pt
\end{center}
\vskip5pt
Das Prinzip ist, dass man eine partikuläre Lösung $y_p(t)$ findet, die die Diff'gleichung sicher erfüllt. Die allgemeine Lösung ist dann $y(t) = y_h(t) + y_p(t)$ \par
\vskip6pt

\textbf{Vorgehen:} \vskip1pt

\begin{enumerate}[label=\protect\circled{\arabic*}]
\item Man nimmt an, dass die partikuläre Lösung $y_p(t)$ konstant ist. Daraus folgt, dass $y_p'(t) = 0$. Man löse also das Gleichungssystem $A \cdot y_p = -b$

\item Man addiere die homogene und die Partikuläre Lösung zusammen: $y(t) = y_h(t) + y_p(t)$

\end{enumerate}

}
\WhiteSpace