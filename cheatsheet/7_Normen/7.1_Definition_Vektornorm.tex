\subsection{7.1 Definition Vektornorm}{
\vskip1pt
Eine Norm im Vektorraum $V$ ordnet jedem Vektor $v$ eine relle Zahl $||v||$ zu und kann so als eine Art Mass verstanden werden. \par
\vskip5pt

Sie muss folgende Bedingungen erfüllen:
\vspace{-1pt}
\begin{center}
\begin{minipage}[t]{0.7 \columnwidth}
\begin{enumerate}[label=\protect\circled{\arabic*}]
\item $||v|| \geq 0$ und $||v|| = 0 \Leftrightarrow v = 0$
\item $||\alpha \cdot v|| = |\alpha|\cdot||v||$
\item $||v + w|| \leq ||v|| + ||w||$
\end{enumerate}
\end{minipage}
\end{center}

}
\WhiteSpace
