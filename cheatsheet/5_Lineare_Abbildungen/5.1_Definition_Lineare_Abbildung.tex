\subsection{5.1 Definition Lineare Abbildung}{
\vskip1pt
Eine Abbildung $\mathcal{F}$ heisst \textbf{linear}, falls $\forall x, y \in V, \forall \alpha \in \mathbb{K}$
\vskip8pt
\begin{center}
\begin{minipage}[t]{0.6 \columnwidth}
\begin{enumerate}[label=\protect\circled{\arabic*}]
\item $\mathcal{F}(x + y) = \mathcal{F}(x) + \mathcal{F}(y)$
\item $\mathcal{F}(\alpha \cdot x) = \alpha \cdot \mathcal{F}(x)$
\end{enumerate}
\end{minipage}
\end{center}
\vspace{0pt}

\begin{itemize}[leftmargin=0.29cm, itemsep=0.5pt]
\item Eine Abbildung ist linear $\Longrightarrow$ \textbf{bildet 0 auf 0 ab!}
\item Eine Abbildung zwischen endlichdimensionalen VR ist linear $\Longleftrightarrow$ kann mit einer $m \times n$-Matrix A mit Hilfe der Matrizenmultiplikation dargestellt werden.
\end{itemize}
\vspace{-5pt}
}
\WhiteSpace
