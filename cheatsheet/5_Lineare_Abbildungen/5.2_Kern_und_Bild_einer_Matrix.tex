\subsection{5.2 Kern und Bild einer Matrix}{
\vskip1pt
\textbf{Kern:}
Der Kern einer Matrix ist die Menge aller Vektoren, die durch Multiplikation auf den Nullvektor abgebildet werden.
\vskip3pt
\begin{center}
$Kern(A) = \{x \in \mathbb{R}^n | A\cdot x = 0 \}$
\end{center}

\begin{itemize}[leftmargin=0.29cm, itemsep=0.5pt]
\item \textbf{Kern bestimmen:} Gleichungssystem $Ax = 0$ lösen. \par Wenn Rang nicht voll ist, gibt es unendlich viele Lösungen. Der Lösungsraum (am besten dargestellt als Linearkombination von mit Parameter multiplizierten Vektoren) ist der Kern der Matrix.
\end{itemize}

\vskip2pt

\textbf{Bild:}
Das Bild einer Matrix A ist die Menge aller Bildvektoren, also aller möglichen \glqq Ergebnisse\grqq einer Multiplikation von A mit einem beliebigen Vektor.

\vskip3pt
\begin{center}
$Bild(A) = \{y \in \mathbb{R}^m | \hskip2pt \exists x \in \mathbb{R}^n$ sodass $y = A\cdot x \}$
\end{center}
\vspace{-5pt}

\begin{itemize}[leftmargin=0.29cm, itemsep=0.5pt]
\item \textbf{Bild bestimmen:} $Bild = span\{a^{(1)}, a^{(2)} \hdots a^{(n)}\}$ \par
Ist nach einem Erzeugendensystem gefragt, reicht es, einfach die Spaltenvektoren hinzuschreiben. \par
\textbf{Achtung:} Die Spaltenvektoren sind immer ein Erzeugendensystem des Bildes, jedoch nicht unbedingt eine Basis! Um Basis zu erstellen: Siehe 4.5
\end{itemize}

\vspace{0pt}

\textbf{Zusammenhänge}
\begin{itemize}[leftmargin=0.29cm, itemsep=0.5pt]
\item $dim(Bild(A)) = Rang(A)$
\item für $A^{m \times n}$: $dim(Bild(A)) + dim(Kern(A)) = n$
\item $Bild(A) \perp Kern(A^T)$
\item $Bild(A^N) \subseteq Bild(A)$ und es gibt $A$ derart, dass $Bild(A^N) \neq Bild(A)$ mit ($A \in \mathbb{R}^{n \times n}$)
\item Fredholm Alternative: $Ax = b$ ist lösbar (b liegt im Bild) genau dann, wenn $b$ senkrecht auf allen
Lösungen des adjungierten LGS $A^T \cdot y = 0$ steht.
\end{itemize}

}
\WhiteSpace
