\subsection{5.3. Abbildungsmatrix aus gegebener Abbildung}{
\vskip1pt
Idee: Wir bilden zuerst die Basisvektoren ab und konstruieren uns aus den Ergebnissen unsere Matrix. \par
\vspace{0pt}
\begin{center}Bsp: $P_2 \rightarrow P_1: p(x) \mapsto p'(x)$ \end{center}
\vspace{-5pt}
\begin{enumerate}[label=\protect\circled{\arabic*}]
\item Finde Basis für Vektorraum aus dem man abbildet und für Vektorraum in den man abbildet. \par
\vskip2pt Bsp: Basis für $P_2 = \{x^2, x, 1\}$ \par
\hskip16pt Basis für $P_1 = \{x, 1\}$
\item Überlege, was nach gegebener Abbildungsvorschrift mit den Basisvektoren passiert und schreibe die Ergebnisse in Vektorschreibweise. \par
\vskip2pt Bsp: $(x^2)' = 2\cdot x = \begin{pmatrix} 2 & 0 \end{pmatrix}^T$ \par
\hskip17pt $(x)' = 1 = \begin{pmatrix} 0 & 1 \end{pmatrix}^T$ \par
\hskip17pt $(1)' = 0 = \begin{pmatrix} 0 & 0 \end{pmatrix}^T$
\item Resultate von Punkt 2 sind Spalten der gesuchten Matrix. (Multiplikation mit Basisvektor = Extraktion von Spalte)
\vskip2pt Bsp: $A = \begin{pmatrix} 2 & 0 & 0 \\ 0 & 1 & 0 \end{pmatrix}$
\end{enumerate}
\vspace{-1mm}
}
\WhiteSpace
