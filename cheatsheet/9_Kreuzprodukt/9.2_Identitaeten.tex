\subsection{9.2 Identitäten}{
\vskip1pt

Mithilfe der folgenden Identitäten können verschiedene Ausdrücke ineinander übergeführt werden.\par \vskip7pt

\textbf{Jacobi-Identität}\par

\vspace{-1pt}
\begin{center}
$a \times(b \times c) +b \times (c \times a) +c \times (a \times b) = 0$
\end{center}\vskip3pt

\textbf{Graßmann-Identität}\par
\vspace{-1pt}
\begin{center}
$a \times(b \times c) = (a \cdot c)\, b - (a \cdot b)\, c$\vskip2pt
$(a \times b) \times c = (a \cdot c)\, b - (b \cdot c)\, a$
\end{center}\vskip3pt

\textbf{Lagrange-Identität}\par
\vspace{-1pt}
\begin{center}
$(a \times b) \cdot (c \times d) = (a \cdot c) (b \cdot d) - (b \cdot c) (a \cdot d)$
\end{center}\vspace{-3pt}

}
\colbreak
%\WhiteSpace
