\subsection{13.2 Givens-Rotation}{
\vskip1pt

Die Givens-Rotation ermöglicht eine Rotation um eine fixe Achse in $\mathbb{R}^3$ mit Orthonormalbasis.\par

\begin{center}
$U_x(\phi) := \begin{pmatrix}
				      1       &       0       &      0       \\
			          0       &  \cos(\alpha) & -\sin(\alpha)\\
				      0       & \sin(\alpha)  &  \cos(\alpha) 
			  \end{pmatrix}$\vskip3pt\par
$U_y(\phi) := \begin{pmatrix}
			     \cos(\alpha) &       0       &  \sin(\alpha) \\
				       0      &       1       &      0        \\
				-\sin(\alpha) &       0       &  \cos(\alpha)
			  \end{pmatrix}$\vskip3pt\par
$U_z(\phi) := \begin{pmatrix}
			     \cos(\alpha) & -\sin(\alpha) &      0       \\
				 \sin(\alpha) &  \cos(\alpha) &      0       \\
				      0       &       0       &      1
			  \end{pmatrix}$\par	  
\end{center}

\textbf{Charakteristisches Polynom}\par\vskip1pt
Sie besitzt das charakteristische Polynom
\begin{center}
$(1 - \lambda)(\lambda^2 -2\lambda cos(\alpha) + 1) = 0$
\end{center}
\vskip3pt

\textbf{Eigenwerte}\par\vskip2pt
Und die Eignewerte ergeben sich wie folgt

\vspace{-7pt}
\begin{align*}
\lambda_1 &= 1 \\
\lambda_2 &= cos(\alpha) + isin(\alpha) \\
\lambda_3 &= cos(\alpha) - isin(\alpha)
\end{align*}
\vspace{-7pt}

Dabei ergeben sich die Eigenvektoren zu
\begin{center}
$v_1 = \begin{pmatrix} 1\\ 0\\ 0\end{pmatrix}, \qquad v_2 = \begin{pmatrix} 0\\ i\\ 0\end{pmatrix}, \qquad v_3 = \begin{pmatrix} 0\\ -i\\ 0\end{pmatrix}$
\end{center}

}
\WhiteSpace