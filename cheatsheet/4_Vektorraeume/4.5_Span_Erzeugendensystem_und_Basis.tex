\subsection{4.5 Span, Erzeugendensystem und Basis}{
\vskip3pt

Die \textbf{lineare Hülle} $span(v_1, v_2, \dotsm, v_n)$ ist die Menge aller endlichen Linearkombinationen der $v_i$ mit Skalaren aus $\mathbb{R}$. 
\par
\vskip3pt
Falls für einen Vektorraum gilt $span(v_1, v_2, \dotsm, v_n) = V$, heisst $\{ v_1, v_2, \dotsm, v_n\}$ ein \textbf{Erzeugendensystem} von V. \par
\vskip3pt
Falls ein Erzeugendensystem für V aus linear unabhängigen Vektoren besteht, heisst es \textbf{Basis} von V. Jeder Vektor kann \textbf{eindeutig} als Linearkombination von Basisvektoren dargestellt werden.

\vskip8pt
\textbf{Aus Erzeugendensystem Basis finden:}
\begin{enumerate}[label=\protect\circled{\arabic*}]
\item Matrix aufstellen, deren Zeilen aus den transponierten erzeugenden Vektoren besteht.
\item Mit Gaussalgorithmus in Zeilenstufenform bringen. Dadurch wird lineare Abhängigkeit eliminiert.
\item Die verbleibenden Nicht-Nullzeilen sind Basisvektoren.
\end{enumerate}
\vspace{-3pt}
}
%\WhiteSpace