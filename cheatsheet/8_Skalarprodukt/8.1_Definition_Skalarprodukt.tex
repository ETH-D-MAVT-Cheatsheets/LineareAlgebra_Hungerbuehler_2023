\subsection{8.1 Definition Skalarprodukt}{
\vskip1pt
Ein Skalarprodukt ordnet jedem Paar $x$, $y$ von Vektoren eine Zahl $\langle x, y \rangle$ zu. \par
\vskip5pt

Es muss folgende Bedingungen erfüllen:
\vspace{-1pt}
\begin{center}
\begin{minipage}[t]{0.8 \columnwidth}
\begin{enumerate}[label=\protect\circled{\arabic*}]
\item $\langle x, y + z \rangle = \langle x, y \rangle + \langle x, z \rangle$
\item $\langle x, \alpha y \rangle = \alpha \langle x, y \rangle$
\item $\langle x, y \rangle = \langle y, x \rangle$
\item $\langle x, x \rangle \geq 0$ und $\langle x, x \rangle = 0 \Leftrightarrow x = 0$
\end{enumerate}
\end{minipage}
\end{center}


\vskip4pt
\textbf{Beispiele für Skalarprodukte}
\begin{itemize}[leftmargin=0.29cm, itemsep=0.5pt]
\item Standardskalarprodukt auf $\mathbb{R}^n$: $\langle x, y \rangle = x^T \cdot y$
\item Funktionenskalarprodukt: $\langle f, g \rangle = \int_{a}^{b}f(x) g(x) dx$
\end{itemize}

\vskip2pt
\textbf{Rechenregeln:} \par
\vspace{-3mm}

\hskip5pt
\begin{minipage}[t]{\columnwidth}
\begin{align}
\langle Ax, Ay \rangle &= \langle x, A^TAy \rangle \nonumber \\
cos\phi &= \frac{\langle a, b \rangle}{\|a\| \|b\|} \qquad \forall\: a \land b \in \mathbb{R}^2 \nonumber
\end{align}
\end{minipage}


}
\WhiteSpace
