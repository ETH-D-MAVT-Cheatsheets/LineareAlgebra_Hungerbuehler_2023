\subsection{8.2 Von Skalarprodukt induzierte Norm}{
\vskip1pt
Aus einem Skalarpodukt kann eine Norm induziert werden. Dieser Ausdruck erfüllt alle Axiome für eine Norm (siehe 7.1.)
\vskip2pt
\begin{center}
$||x|| = \sqrt{\langle x, x \rangle}$
\end{center}

Eine Norm $\| \cdot \|$ wird genau dann von einem Skalarprodukt induziert, wenn die \textbf{Parallelogrammregel} gilt:
\vskip2pt
\begin{center}
$\|x + y \|^2 + \|x - y \|^2 = 2(\|x\|^2 + \|y\|^2)$
\end{center}

In diesem Fall ist das Skalarprodukt durch die \textbf{Polarisationsformel} aus der Norm rekonstruierbar:
\vskip2pt
\begin{center}
$\langle x,y \rangle = \frac{1}{4}(\|x + y \|^2 - \|x - y \|^2)$
\end{center}

}
\WhiteSpace