\subsection{2.1 Matrixschreibweise}{
	
\vskip1pt
Ein lineares Gleichungssystem kann in Matrixschreibweise dargestellt werden:

\vspace{-1mm}
\begin{center}
\begin{minipage}[t]{0.33\columnwidth}
\renewcommand{\arraystretch}{1.1}
  \begin{tabular}{C{2mm}  C{2.5mm} | C{2mm}}
  $x_1$ & $x_2$ & 1 \\ \hline
  $a_{11}$ & $a_{12}$ & $b_1$  \\
  $a_{21}$ & $a_{22}$ &  $b_2$  \\
  $a_{31}$ & $a_{32}$ &  $b_3$  \\
  \end{tabular}
\end{minipage}
\begin{minipage}[t]{0.12 \columnwidth}
$\Longleftrightarrow$
\end{minipage}
\begin{minipage}[t]{0.5 \columnwidth}
$\begin{pmatrix}
  a_{11} & a_{12} \\
  a_{21} & a_{22} \\
  a_{31} & a_{32} \\
\end{pmatrix}$
$\cdot$
$\begin{pmatrix}
x_1 \\
x_2
\end{pmatrix}$
$=$
$\begin{pmatrix}
b_1 \\
b_2 \\
b_3
\end{pmatrix}$
\end{minipage}
\end{center}
}
\WhiteSpace