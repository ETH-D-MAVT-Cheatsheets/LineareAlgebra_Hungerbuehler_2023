\subsection{2.7 Orthogonale Matrizen}{


Eine quadratische Matrix ist orthogonal, wenn sie aus \textbf{zueinander orthogonalen Spaltenvektoren der Länge 1} besteht. Dies ist der Fall, wenn eine Matrix mit der Transponierten multipliziert die Identität ergibt.

%\vspace{-5pt}

\begin{center}
\fbox{$Q^T \cdot Q = I$}
\end{center}

Multipliziert man einen Vektor mit einer orthogonalen Matrix, kann sich seine Orientierung ändern, jedoch nicht seine Länge. Abbildungen sind deshalb kongruent.

\vskip5pt

\textbf{Eigenschaften:}

\vspace{-3pt}

\begin{itemize}[leftmargin=0.29cm, itemsep=0.5pt]
\item $Q$ orthogonal $\Leftrightarrow$ Spalten/Zeilen von $Q$ sind zueinander orthogonale Vektoren der Länge 1.
\item Nur quadratische Matrizen können orthogonal sein.
\item $A$ und $B$ orthogonal $\Rightarrow$ $A \cdot B$ orthogonal
\item $Q$ orthogonal $\Leftrightarrow$ $Q^{-1}$ orthogonal
\item $\|Qx\|_2 = \|x\|_2$, $|det(Q)| = 1$, $|eig(Q)| = 1$
\item $Q^{-1} = Q^T$
\end{itemize}
\vspace{-5pt}

%\textbf{Drehungs- und Spiegelungsmatrizen:}
%
%\vskip1pt
%
%Drehung um Ursprung im $\mathbb{R}^2$: \par \vspace{-1mm} \begin{center}$R(\alpha) = \begin{pmatrix} cos(\alpha) & - sin(\alpha) \\ sin(\alpha) & cos(\alpha) \end{pmatrix}$\end{center} \par
%Drehung um x-Achse im $\mathbb{R}^3$: \par \vspace{-1mm} \begin{center}
%$R_x(\alpha) = \begin{pmatrix} 1 & 0 & 0 \\ 0 & cos(\alpha) & -sin(\alpha) \\ 0 & sin(\alpha) & cos(\alpha)  \end{pmatrix}$ \end{center}
%Drehung um y-Achse im $\mathbb{R}^3$: \par \vspace{-1mm} \begin{center}
%$R_y(\alpha) = \begin{pmatrix} cos(\alpha) & 0 & sin(\alpha) \\ 0 & 1 & 0 \\ -sin(\alpha) & 0 & cos(\alpha)  \end{pmatrix}$ \end{center}
%Drehung um z-Achse im $\mathbb{R}^3$: \par \vspace{-1mm} \begin{center}
%$R_z(\alpha) = \begin{pmatrix} cos(\alpha) & -sin(\alpha) & 0 \\ sin(\alpha) & cos(\alpha) & 0 \\ 0 & 0 & 1  \end{pmatrix}$ \end{center}
%
%
}
\WhiteSpace